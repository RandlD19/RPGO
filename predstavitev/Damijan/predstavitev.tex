\documentclass[12pt]{article}
\usepackage[utf8]{inputenc}
\usepackage[slovene]{babel}
\usepackage{amsthm,amsfonts,amsmath,amssymb,url}
\usepackage{hyperref}

\begin{document}
\section*{Kubične PH krivulje}
...
V nadaljevanju se bomo posvetili le PH krivuljam kjer je $w(t)=1$, polinoma $u$ in $v$ pa nimata skupnih ničel - sta si tuja.
Hodografu s temi lastnostmi pravimo primitiven pitagorejski hodograf in definira t.i. regularno PH krivuljo (saj velja $r'(t) \neq 0 \forall t$).

Poglejmo si stopnjo takih krivulj. V prejšni opombi smo ugotovili da so PH krivulje stopnje $n+2m+1$, kjer je $n$ predstavljala stopnjo polinoma $w$.
Ta pa je sedaj enak $1$, zato se stopnja poenostavi v $2m + 1$. PH krivulje s primitivnim pitagorejskim hodografom so torej lihe stopnje. 
Ker pa je PH krivulje 1. stopnje kar premica, kar ni najbolj zanimivo se bomo osredotočili na kubične PH krivulje.

Konstrukcijo take PH krivulje bomo začeli z definiranjem $u(t)$ in $v(t)$, ki jih zapišemo v Bernsteinovi bazi na intervalu $[0,1]$.
$$ \textbf{u}(t) = u_0 B_0^1(t) + u_1 B_1^1(t) $$
$$ \textbf{v}(t) = v_0 B_0^1(t) + v_1 B_1^1(t) $$
Pri tem upoštevamo, da $u$ in $v$ nimata skupnih ničel, $w$ pa je konstanta.
Z upoštevanjem teh dveh zvez in pa prej navedenih zvez hodografa dobimo naslednji komponenti hodografa:
$$\textbf{x}'(t) = (u_0^2 - v_0^2)B_0^2(t) + (u_0 u_1 - v_0 v_1) B_1^2(t) + (u_1^2 - v_1^2) B_2^2(t),$$
$$\textbf{y}'(t) = 2 u_0 v_0 B_0^2(t) + (u_0 v_1 + u_1 v_0) B_1^2(t) + 2 u_1 v_1 B_2^2(t).$$
Če želimo pridobiti sedaj kontrolne točke takšne krivulje, pa moramo ta pridobljen hodograf integrirati.
Do rezultata bi lahko prišli kar z uporabo definicije Bernstainovih baznih polinomov, a si vseeno na hitro poglejmo še formulo za nedločen integral Bernstainovega polinoma.
Ta se glasi:
$$\int B^n_i(t) dt = \frac{1}{n+1} \sum_{j=i+1}^{n+1} B^{n+1}_j(t)$$
Tu vidimo, da spredaj pridobimo se neko konstanto in sicer v našem primeru $\frac{1}{3}$, ki jo lahko opazimo sedaj pri definiciji kontrolnih točk B\'ezierjeve PH krivulje:
% \begin{eqnarray}
%     \textbf{b}_1 &=& \textbf{b}_0 + \frac{1}{3}(u_0^2 - v_0^2, 2 u_0 v_0)^T,\nonumber\\
%     \textbf{b}_2 &=& \textbf{b}_1 + \frac{1}{3}(u_0 u_1 - v_0 v_1, u_0 u_1 + v_0 v_1)^T,\nonumber\\
%     \textbf{b}_3 &=& \textbf{b}_2 + \frac{1}{3} (u_1^2 - v_1^2, 2 u_1 v_1)^T,\nonumber
% \end{eqnarray}
Pri tem točka $b_0$ ustreza konstantam pri integraciji. 

\section*{Racionalni odmik PH krivulje}
Še ena pomembna lastnost PH krivulj, ki sma jih omenila na začetku pa je, da so odmiki PH krivulje oziroma paralelne krivulje, racionalne krivulje. 
Za začetek si poglejmo splošno definicijo paralelne krivulje $r_d$, ki je od $r$ oddaljena za $d$. Ta je definirana kot:
$$\textbf{r}_d(t) = \textbf{r}(t) + d \textbf{n}(t)$$
Če je $r$ B\'ezierjeva PH krivulja, potem je $r_d$ racionalna B\'ezierjeva krivulja in je stopnje 5 za kubične PH krivulje. 
To pa sledi ravno iz dejstva, da je normala $n(t)$, ki nastopa v definiciji in jo je ravno prej predstavil Tit podana kot racionalna funkcija. 
Paralelno krivuljo $r_d(t)$ lahko tako izrazimo kot 
$$ \textbf{r}_d(t) = \left(\frac{X(t)}{W(t)}, \frac{Y(t)}{W(t)}\right)$$ 
Koordinate kontrolnih točk racionalne paralelne krivulje pa lahko predstavimo s kontrolnimi točkami originalne krivulje s pomočjo formule:
$$\textbf{O}_k = \sum_{j=\max(0, k-n)}^{\min(n-1, k)} \frac{\binom{n-1}{j} \binom{n}{k-j}}{\binom{2n-1}{k}}(\sigma_j \textbf{P}_{k-j}+ d n \Delta\textbf{P}^\perp_j),$$ 
$$k=0, ..., 2n-1$$
Pri tem so $P_k = (W_k, X_k, Y_k),    k = 0, ..., 2n-1$ kontrolne točke podane krivulje $r$ v homogenih koordinatah. 
Razlike $\Delta P_k$ pa so definirane kot $\Delta P_k = P_{k+1} - P_k = (0, \Delta x_k, \Delta y_k)$
Označimo pa še $\Delta P_k^{\perp} = (0, -\Delta y_k, \Delta x_k)$

\end{document}