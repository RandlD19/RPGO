\documentclass[12pt]{beamer}
\usepackage[slovene]{babel}
\usepackage{lmodern}
\usepackage[T1]{fontenc}
\usepackage[utf8]{inputenc}
\usepackage{amssymb,amsmath}

\title{Ravninske krivulje s pitagorejskim hodogramom}
\author{Tit Arnšek, Damijan Randl}
\date{Januar 2023}
\institute[Inst.]{Fakulteta za matematiko in fiziko \\ Predmet: Geometrijsko podprto računalniško oblikovanje}
\date{Januar 2023}

\begin{document}

\begin{frame}
    \titlepage
\end{frame}

\begin{frame}
\begin{enumerate}
\item Ravninske krivulje s pitagorejskim hodografom
\item B\'ezierjeve kontrolne točke krivulj s PH
\item Parametrična hitrost in dolžina loka
\item Odvod krivulje
\item Racionalni odmiki krivulj s PH
\end{enumerate}
\end{frame}

\begin{frame}
    \frametitle{Kubične PH krivulje}
    \begin{eqnarray}
        x\prime (t) &=& (u^2_0- v^2_0)B^2_0(t) +\nonumber\\
        & & (u_0u_1-v_0v_1)B^2_1(t)+(u^2_1-v^2_1)B^2_2(t),\nonumber\\
        y\prime(t) &=& 2u_0v_0B^2_0(t)+(u_0v_1+u_1v_0)B^2_1(t)+2u_1v_1B^2_2(t).\nonumber
    \end{eqnarray}
    \begin{block}{}
        \begin{eqnarray}
            \textbf{p}_1 &=& \textbf{p}_0 + \frac{1}{3}(u_0^2-v_0^2,2u_0v_0),\nonumber\\
            \textbf{p}_2 &=& \textbf{p}_1 + \frac{1}{3}(u_0u_1-v_0v_1,u_0v_1+u_1v_0),\nonumber\\
            \textbf{p}_3 &=& \textbf{p}_2 + \frac{1}{3}(u_1^2-v_1^2,2u_1v_1),\nonumber
        \end{eqnarray}
    \end{block}
\end{frame}

\end{document}