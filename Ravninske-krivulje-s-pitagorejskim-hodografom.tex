\documentclass[12pt]{article}
 \usepackage[utf8]{inputenc}
   \usepackage[slovene]{babel}
\usepackage{amsthm,amsfonts,amsmath,amssymb,url}
\usepackage{hyperref}

\textheight 210 true mm
\textwidth 146 true mm
\voffset=-17mm
\hoffset=-13mm

\newtheorem{Izrek}{{\sc Izrek}}[section]
\newtheorem{Trditev}[Izrek]{{\sc Trditev}}
\newtheorem{Lema}[Izrek]{{\sc Lema}}
\newtheorem{Posledica}[Izrek]{{\sc Posledica}}
\newtheorem{Definicija}[Izrek]{{\sc Definicija}}
\newtheorem{Domneva}[Izrek]{{\sc Domneva}}
\newtheorem{Zgled}[Izrek]{{\sc Zgled}}
\newtheorem{Opomba}[Izrek]{{\sc Opomba}}

\def\theIzrek{{\rm \arabic{section}.\arabic{Izrek}}}

\newenvironment{izrek}{\begin{Izrek}\sl}{\end{Izrek}}
\newenvironment{trditev}{\begin{Trditev}\sl}{\end{Trditev}}
\newenvironment{lema}{\begin{Lema}\sl}{\end{Lema}}
\newenvironment{posledica}{\begin{Posledica}\sl}{\end{Posledica}}
\newenvironment{definicija}{\begin{Definicija}\rm }{\end{Definicija}}
\newenvironment{domneva}{\begin{Domneva}\rm }{\end{Domneva}}
\newenvironment{zgled}{\begin{Zgled}\rm}{\end{Zgled}}
\newenvironment{opomba}{\begin{Opomba}\rm}{\end{Opomba}}

\newenvironment{dokaz}[1][{\sc Dokaz}]{\begin{proof}[#1]\renewcommand*{\qedsymbol}{\(\square\)}}{\end{proof}}

\newcommand{\Mod}[1]{\hbox{ (mod } #1)}

\begin{document}
% UVOD - ??
\section{Uvod}
Polinomske krivulje s pitagorejskim hodografom (PH krivulje) sta prvič predstavila Farouki in Sakkalis leta 1990.
Karakterizirane so z lastnostjo, da je njihova parametrična hitrost oziroma odvod ločne dolžine v odvisnosti od parametra,
polinomska funkcija. Polinomske PH krivulje tvorijo pomemben razred parametričnih polinomskih krivulj
za katere velja, da je njihova ločna dolžina lahko določena eksaktno in da so njeni odmiki (paralelne krivulje) racionalne krivulje.
Zaradi naštetih lastnosti so zelo uporabne v sistemih CAD/CAM, robotiki, animacijah, NC machining??, spatial path planning based on rotation-minimizing frames?????...

% RAVNINSKE KRIVULJE S PITAGOREJSKIM HODOGRAFOM - TIT 

Hodograf parametično podane odvedljive krivulje $r = (x(t), y(t))^T$ je krivulja, 
podana s predpisom $r' = (x', y')^T$, pravzaprav gre za odvod podane krivulje. 
Krivulje s pitagorejskim hodografom so polinomske parametrične krivulje, 
ki zadoščajo naslednjemu pogoju:
$x'^2 + y'^2 = ro^2$ za nek polinom $ro^2$.
Krajše jih imenjujemo tudi PH krivulje. 
Zaradi lepih lastnosti jih pogosto uporabljamo v...

Za izpeljavo lastnosti v nadaljevanju, bomo potrebovali naslednji izrek:
Izrek (Kubota)
Polinomi a, b in c zadostijo pitagorejskemu pogoju
$$a^2(t) + b^2(t) = c^2(t),$$
natanko tedaj, ko jih lahko izrazimo z drugimi polinomi $u(t), v(t) in w(t)$ v obliki
$$a(t) = (u^2(t) - v^2(t))w(t),$$
$$b(t) = 2uvw,$$
$$c(t) = (u^2 + v^2)w,$$
kjer u in v nimata paroma skupnih ničel
Dokaz: ...

Opombe:
\begin{itemize}
	\item Če velja, da je $w=0$ ali pa je $u=v=0$, potem taka PH krivulja $r(t)$ predstavlja točko.
	\item Če so u, b in w konstante in $w \neq 0$, ter vsaj eden od polinomov u ali v neničelna konstanta,
	potem je PH krivulja enakomerno parametrizirana daljica.
	\item Če sta u in v konstanti, w pa ni konstanta je PH krivulja neenakomerno parametrizirana daljica ali premica.
	\item Če je $w = 0$ in $u = +-v$ ali pa je vsaj eden od u, v ničelni polinom, 
	potem je PH krivulja neenakomerno parametrizirana daljica.
\end{itemize}

Opomba:
Če je w polinom stopnje n in z m označimo večjo izmed stopenj polinomov u in v, 
potem je PH krivulja dobljena z integracijo hodografa stopnje k = n + 2m + 1

% BEZIERJEVE KONTROLNE TOČKE KRIVULJ S PH - DAMIJAN
\section{B\'ezierjeve kontrolne točke PH krivulj}
V nadaljevanju se bomo osredotočili na hodografe kjer je $w(t) = 1$ in je $GCD(u,v)$ konstanta.
Imenjujemo jih primitivni Pitagorejski hodografi. 
Takšni hodografi definirajo regularne PH krivulje, ki zadostujejo pogoju $r'(t) \neq 0$ za vsak $t$. 
Iz opombe (stevilka) sledi, da je stopnja PH krivulje, definirane z integracijo
primitivnih hodografov, lihe stopnje $(k = 2m + 1)$.

Karakterizacijo PH krivulj želimo izraziti v Bezierjevi obliki.

Oglejmo si primer primitivne PH krivulje 3. stopnje v Bezierjevi obliki:
Zapišimo $u$ in $v$ v Bernsteinovi bazi in pri tem upoštevamo pogoj $w(t) = 1$
$$u(t) = u_0 B^1_1(t), v(t) = v_0 B^1_0(t) + v_1 B^1_1(t)$$
Ker sta $u$ in $v$ tuja, sledi, da mora biti $u_0*v_1 \neq u_1*v_0$. Iz dejstva, da
$u$ in $v$ nista obe konstanti, pa sledi $(u_0 - u_1)^2 + (v_0 - v_1)^2 \neq 0$.

Z upoštevanjem zgornjega izreka pa dobimo naslednja predpisa za $x'(t)$ in $y'(t)$
$$x'(t) = (u_0^2 - v_0^2) B_0^2(t) + (u_0 u_1 - v_0 v_1) B_1^2(t) + (u_1^2 - v_1^2) B_2^2(t),$$
$$y'(t) = 2 u_0 v_0 B_0^2(t) + (u_0 v_1 + u_1 v_0) B_1^2(t) + 2 u_1 v_1 B_2^2(t).$$

Definirajmo sedaj (brez dokaza) še pravilo za integriranje Bernstainovih baznih polinomov:
Z integracijo dobimo kubično Bezierjevo krivuljo podano s predpisom:
$$p(t) = b_0*B_0^3(t) + b_1*B_1^3(t) + b_2*B_2^3(t) + b_3*B_3^3(t)$$

kjer je
$$b_1 = b_0 + \frac{1}{3}*(u_0^2 - v_0^2, 2*u_0*v_0)^T,$$
$$b_2 = b_1 + \frac{1}{3}*(u_0*u_1 - v_0*v_1, u_0*u_1 + v_0*v_1)^T,$$
$$b_3 = b_2 + \frac{1}{3}*(u_1^2 - v_1^2, 2*u_1*v_1)^T,$$

Pri tem je $p_0$ prosto izbrana kontrolna točka (zaradi konstante pri integraciji).

% PARAMETRIČNA HITROST IN DOLŽINA LOKA - TIT
\section*{Parametrična hitrost in dolžina loka}
parametrična hitrost regularne PH krivulje $r(t) = (x(t), y(t))$ je podana s predpisom:
$$ | r\prime (t) | =\sqrt{x\prime^2(t)+y\prime^2(t)}= u^2 (t) + v^2 (t) = \sigma (t),$$
in je polinom v $t$. 
Iz lastnosti, da je krivulja, ki jo dobimo z integrinjanjem stopnje (k = 2m + 1) sledi
da morata biti u in v stopnje  $m = \frac{1}{2}(n - 1)$. Zapišemu ju lahko v 
Bernstainovi bazi:

	$u (t)=\sum_{k=0}^m u_kB_k^m(t)$, 
	$v (t) =\sum_{k=0}^m v_kB_k^m(t).$

Potem je 
$$\sigma (t) =\sum_{k=0}^{n-1} \sigma_kB_k^{n-1}(t),$$, 
kjer so koeficienti 
$$\sigma_k =\sum_{j=max(0,k-m)}^{min(m,k)}\frac{\binom{m}{j}\binom{m}{k-j}}{\binom{n-1}{k}}(u_ju_{k-j}+v_jv_{k-j}),$$ $$k = 0,\ldots , n - 1.$$

Če pogledamo primer od zgoraj(kubično B. krivuljo) dobimo naslednje koeficiente za sigmo :

$\sigma_0 = u^2_0+ v^2_0, $
$\sigma_1 = u_0u_1 + v_0v_1,$ 
$\sigma_2 = u^2_1+ v^2_1.$

Poglejmo si sedaj še dolžino loka s.
To dobimo kot:
$$s (t) =\int^t_0\sigma(\tau) d\tau,$$.
Pri računanju integrala upoštevamo integracijsko pravilo za B. bazne polinome 
DOPIŠI PRAVILO ZDAJ ALI PA ŽE PREJ KO GA UPORABIMA PRI ENI IZPELJAVI
Zgornji izraz se nam poenostavi v 
$$s (t) =\sum^n_{k=0}s_k\binom{n}{k}(1-t)^{n-k}t^k=\sum_{k=0}^n s_kB^n_k(t),$$
	kjer je $s_0=0$ in $s_k=\frac{1}{n}\sum^{k-1}_{j=0}\sigma_j, k=1,\ldots,n.$.

Skupna dolžina loka je tako S(1). 

Običajno smo za  b. krivulje prikazali z vrednotenjem vrednosti parametrov  $t_0,\ldots , t_N$,
(enakomerni razmik med parametri )... 
ampak smo s tem dobili neenakomerno porazdeljene točke na krivulji, saj parametrična 
hitrost, v splošnem, ni konstantna.
Če želimo doseči, da bodo točke $s(t_k)$ vseeno enakomerno porazdeljene na grafu,
lahko to dosežemo tako, da za vsak $t_k$ izračunamo nov paramater, ki ga dobimo 
s pomočjo Newton-Raphsonove iteracije: 
$$t^{(0)}_k = t_{k-1}+\frac{\Delta s}{\sigma(t_{k-1})}$$
$$t^{(r)}_k = t^{(r-1)}_k-\frac{s(t^{(r-1)}_k)-k\Delta s}{\sigma(t^{(r-1)}_k)}, r = 1, 2,\ldots.$$

Za dobre rezultate je dovolj že nekaj iteracij. KOOLLIKO JE TO??? (DVA ALI TRI PIŠE V ANGLEŠKEM ČLANKU)

% ODVOD KRIVULJE - TIT 
\section*{Odvod krivulje}
Iz lastnosti odvoda PH krivlja sledi tudi, da so enotski tangentni vektor,
normala in ukrivljenost racionalno odvisni od parametra krivulje. Natancneje, ˇ
definirani so v smislu polinomov u(t) in v(t), in sicer: 

$$\textbf{t} =\frac{(u^2 - v^2, 2uv)}{\sigma},\hspace{10pt} \textbf{n} =\frac{(2uv, v^2 - u^2)}{\sigma},\hspace{10pt} \kappa = 2 \frac{uv\prime - u\prime v}{\sigma^2}.$$

% RACIONALNI ODMIKI KRIVULJ S PH - DAMIJAN 
\section*{Racionalni odmiki PH krivulj}
Racionalni odmiki PH krivulj
Odmik krivulje r(t) je v splošnem definiran kot
$$ r_d(t) = r(t) + dn(t)$$
V primeru, da r(t) pri tem predstavlja Bezierjevo PH krivuljo velja, 
da $r_d(t)$ spada med racionalne Bezierjeve krivulje. V primeru kubične PH krivilje 
je njen odmik 5. stopnje. Normala n je pri tem enaka, kot smo jo definirali zgoraj:
$$\textbf{n} =\frac{(2uv, v^2 - u^2)}{\sigma}$$,
kjer je $\sigma$ parametrična hitrost krivulje r.

Definirajmo kontrolne točke krivulje r v homogenih koordinatah kot
$$ P_k = (W_k, X_k, Y_k) = (1, x_k, y_k), k = 0, ..., n.$$
Z $$\Delta P_k = P_{k+1}-P_k = (0, \Delta x_k, \Delta y_k), k = 0, ..., n-1$$ 
pa definirajmo razlike med njimi. Pri tem so $$\Delta x_k = x_{k+1} - x_k$$ in $$\Delta y_k = y_{k+1} - y_k$$
Označimo še $$ \Delta P_k^{\perp} = (0, \Delta y_k, - \Delta x_k)$$. Potem lahko
racionalni odmik PH krivulje r(t) zapišemo kot
$$ r_d(t) = (\frac{X(t)}{W(t)}, \frac{Y(t)}{W(t)})$$,
kjer so W, X in Y polinomi stopnje $$2n -1$$, katerih koeficienti 
$$ O_k = (W_k, X_k, Y_k), k = 0, ..., 2n-1$$
definirajo kontrolne točke racionalne Bezierjeve krivulje. 
Homogene koordinate kontrolnih točk pa lahko izrazimo tudi s pomočjo kontrolnih točk podane začetne krivulje:
$$ O_k = \sum_{j=max(0,k-n)}^{min(n-1,k)}{\frac{\binom{n-1}{j}\binom{n}{k-j}}{\binom{2n-1}{k}}(\sigma_j P_{k-j}+d n \Delta P^{\perp}_{j})}, k = 0, ..., 2n-1.$$

Kontrolne točke racionalnega odmika kubičnih PH krivulj so tako podane kot:
$$ O_0 = \sigma_0 P_0 + 3 d \Delta P_0^{\perp}$$,
$$ O_1 = \frac{1}{5} [2 \sigma_1 P_0 + 3\sigma_0 P_1 + 3 d (3 \Delta P_0^{\perp} + 2 \Delta P_1^{\perp})]$$,
$$ O_2 = \frac{1}{10} [\sigma_2 P_0 + 6\sigma_1 P_1 + 3\sigma_0 P_2 + 3 d (3 \Delta P_0^{\perp} + 6 \Delta P_1^{\perp}) + \Delta P_2^{\perp})]$$,
$$ O_3 = \frac{1}{10} [3\sigma_2 P_1 + 6\sigma_1 P_2 + \sigma_0 P_3 + 3 d (\Delta P_0^{\perp} + 6 \Delta P_1^{\perp} + 3 \Delta P_2^{\perp})]$$,
$$ O_4 = \frac{1}{5} [3\sigma_2 P_2 + 2\sigma_1 P_3 + 3 d (2\Delta P_1^{\perp} + 3 \Delta P_2^{\perp})]$$,
$$ O_5 = \sigma_2 P_3 + 3 d \Delta P_2^{\perp}$$


\end{document}